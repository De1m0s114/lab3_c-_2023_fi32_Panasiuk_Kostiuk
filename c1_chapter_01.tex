%!TEX root = ../thesis.tex

\chapter{КОМП’ЮТЕРНИЙ ПРАКТИКУМ №1}

\textbf{Мета роботи}: Ознайомлення з пiдходами побудови атак на асиметричнi криптосистеми на прикладi атак накриптосистему RSA, а саме атаки на основi китайської теореми про лишки, що є успiшною при використаннi однакового малого значення вiдкритої експоненти для багатьох користувачiв, та атаки «зу-
стрiч посерединi», яка можлива у випадку, якщо шифротекст є невеликим числом, що є добутком двох
чисел.

\textbf{Постановка задачі}: для заданого варiантом моделi шифру,
реалiзувати атаку з малою експонентою на основi китайської теореми про лишки та Атаку «зустрiч посерединi».

\textbf{Варіант}: 5.

\textbf{Хід роботи}:

1 Реалiзувати атаку з малою експонентою на основi китайської теореми про лишки.

2 Реалiзувати атаку «зустрiч посерединi» та порiвняти її швидкодiю з повним перебором можливих
вiдкритих текстiв.

3 Оформити звiт до комп’ютерного практикуму.


\section{Атака з малою експонентою}

Для проведення цієї атаки було вибрано варіант на 3 системи рівнянь для наших значень(256 біт)

Отриманий результат:

М=1ffffffffffffffff003eeb08d037a852d10f4367210ada8553ed6275bf14

Час роботи 0.001sec


\section{Атака «зустрiч посерединi» }

Для проведення цієї атаки було вибрано варіант на 2048 біт.

Отриманий результат:

М=b1d53

Час роботи 1.54sec

Якщо провести перевірку і піднести $M^e modN=b1d53^{65537}mod N$ і справді отримаємо C.

Будемо вважати за одиницю виміру складності в нашому випадку - обрахування функції RSA.

При повному переборі за наших умов буде $2^20$ таких операцій. 

У випадку атаки ж всього $2*2^{10}$, що у 9 разів швидше.

